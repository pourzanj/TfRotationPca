\RequirePackage{etoolbox}
\csdef{input@path}{%
 {sty/}% cls, sty files
 {img/}% eps files
}%
\csgdef{bibdir}{bib/}% bst, bib files

\documentclass[ba]{imsart}
%
\pubyear{0000}
\volume{00}
\issue{0}
\doi{0000}
\firstpage{1}
\lastpage{1}

%
\usepackage{amsfonts}
\usepackage{blkarray}
\usepackage{amsthm}
\usepackage{amsmath}
\usepackage{natbib}
%\usepackage[colorlinks,citecolor=blue,urlcolor=blue,filecolor=blue,backref=page]{hyperref}
\usepackage{hyperref}
\usepackage{graphicx}

\usepackage{algorithm2e}

\newcommand{\mb}[1]{\mathbf{#1}}
\newcommand{\bs}[1]{\boldsymbol{#1}}
\newcommand{\mvec}[1]{\mathbf{#1}}

\startlocaldefs
\numberwithin{equation}{section}
\theoremstyle{plain}
\newtheorem{thm}{Theorem}[section]
\endlocaldefs

\begin{document}

\begin{frontmatter}
\title{Bayesian Inference over the Stiefel Manifold via the Givens Representation: Appendix}
\runtitle{Appendix: Bayesian Inference via the Givens Representation}

\begin{aug}
\author{\fnms{Arya A. Pourzanjani}\thanksref{addr1}\ead[label=e1]{arya@ucsb.edu}},
\author{\fnms{Richard M. Jiang}\thanksref{addr1}},
\author{\fnms{Brian Mitchell}\thanksref{addr1}},
\author{\fnms{Paul J. Atzberger}\thanksref{addr2}}
\and
\author{\fnms{Linda R. Petzold} \thanksref{addr1}
}

\runauthor{Pourzanjani et al.}

\address[addr1]{Computer Science Department, University of California, Santa Barbara
    \printead{e1} % print email address of "e1"
}

\address[addr2]{Mathematics Department, University of California, Santa Barbara
}

\end{aug}


\end{frontmatter}

%%%%%%%%%%%%%%%%%%%%%%%%%
%%%%%%%%%%%%%%%%%%%%%%%%%
%%%%%%%%%%%%%%%%%%%%%%%%%
\appendix
\section{Deriving the Change-of-Measure Term}
%%%%%%%%%%%%%%%%%%%%%%%%%
%%%%%%%%%%%%%%%%%%%%%%%%%
%%%%%%%%%%%%%%%%%%%%%%%%%
We derive the simplified form of the differential form showing that

\begin{equation}
\label{eq:final_change_of_measure}
\bigwedge_{i=1}^p \bigwedge_{j=i+1}^n G_j^T\, dY_i = \prod_{i=1}^p \prod_{j=i+1}^n \cos^{j-i-1} \theta_{ij}.
\end{equation}

\noindent We point out that \cite{khatri1977mises} provide a similar expression for a slightly different representation, but do not offer a derivation. We start by considering the determinant of the matrix form of the change-of-measure term:

\begin{equation}
\bigwedge_{i=1}^p \bigwedge_{j=i+1}^n G_j^T\, J_{Y_i(\Theta)}(\Theta) d\Theta = 
\begin{pmatrix}
G_{2:n}^T J_{Y_1(\Theta)}(\Theta)\\
G_{3:n}^T J_{Y_2(\Theta)}(\Theta)\\
\vdots\\
G_{p:n}^T J_{Y_p(\Theta)}(\Theta)
\end{pmatrix}
\end{equation}

\noindent For $l = 1, \cdots, n$, we define the following shorthand notation

\begin{equation}
\partial_{i,i+l} Y_k := \frac{\partial}{\partial \theta_{i,i+l}} Y_k
\end{equation}

\noindent and

\begin{equation}
\partial_{i} Y_k
:=
\begin{pmatrix}
\partial_{i,i+1} Y_k & \partial_{i,i+2} Y_k & \cdots & \partial_{in} Y_k.
\end{pmatrix}
\end{equation}

\noindent In the new notation Equation can be written in the following block matrix form:

\begin{equation}
\label{eq:matrix_blockform}
\begin{pmatrix}
G_{2:n}^T \partial_{1} Y_1 &G_{2:n}^T \partial_{2} Y_1 & \cdots & G_{2:n}^T \partial_{p} Y_1\\
G_{3:n}^T \partial_{1} Y_2 &G_{3:n}^T \partial_{2} Y_2 & \cdots & G_{3:n}^T \partial_{p} Y_2\\
\vdots & \vdots & \ddots & \vdots\\
G_{p:n}^T \partial_{1} Y_p &G_{p:n}^T \partial_{2} Y_p & \cdots & G_{p:n}^T \partial_{p} Y_p\\
\end{pmatrix}.
\end{equation}

\noindent Note that the block matrices above the diagonal are all zero.  This can be seen by observing that the rotations in the Givens representation involving elements greater than $i$ will not affect $e_i$, i.e. letting $R_i := R_{i,i+1} \cdots R_{in}$,

\begin{eqnarray}
Y_i = R_1 R_2 \cdots R_p e_i = R_1 \cdots R_i e_i.
\end{eqnarray}

\noindent Thus for $j > i$, $\partial_j Y_i = 0$ and the determinant of Expression \ref{eq:matrix_blockform} simplifies to the product of the determinant of the matrices on the diagonal i.e. the following expression:

\begin{equation}
\label{eq:det_of_blocks}
\prod_{i=1}^p \det \left( G_{i+1:n}^T \partial_{i} Y_i \right).
\end{equation}

%%%%%%%%%%%%%%%%%%%%%%%%%
\subsection{Simplifying Diagonal Block Terms}
%%%%%%%%%%%%%%%%%%%%%%%%%
Let $I_{i}$ denote the first $i$ columns of the $n \times n$ identity matrix and let $I_{-i}$ represent the last $n-i$ columns. The term $G_{i+1:n}^T$ in expression \ref{eq:det_of_blocks} can be written as

\begin{equation}
G_{i+1:n}^T = I_{-i}^T G^T = I_{-i}^T R_p^T \cdots R_1^T.
\end{equation}

\noindent To simplify the diagonal block determinant terms in Expression \ref{eq:det_of_blocks} we take advantage of the following fact

\begin{eqnarray}
\det \left( G_{i+1:n}^T \partial_i Y_i  \right)  &=& \det \left( I_{-i}^T R_p^T \cdots R_1^T \right) =  \det\left( I_{-i}^T R_i^T \cdots R_1^T \partial_i Y_i \right).
\end{eqnarray}

\noindent In other words, the terms $R_p^T \cdots R_{i+1}^T$ have no effect on the determinant. This can be seen by first separating terms so that

\begin{eqnarray}
\det\left(G_{i+1:n}^T \partial_{i} Y_i \right) &=& \det\left( \underbrace{I_{-i}^T}_{(n-i) \times n} R_p^T \cdots R_1^T \underbrace{\partial_i Y_i}_{n \times (n-i)} \right)\\
&=& \det\left(
I_{-i}^T
\left[ R_p^T \cdots R_{i+1}^T\right] \left[ R_i^T \cdots R_1^T \partial_i Y_i \right] \right),
\end{eqnarray}

\noindent and then noticing that $R_{i+1} \cdots R_p$ affects only the first $i$ columns of the identity matrix so 

\begin{eqnarray}
I_{-i}^T
\left[ R_p^T \cdots R_{i+1}^T\right]  &=& \left( R_{i+1} \cdots R_p\, I_{-i} \right)^T = \left( I_{-i} \right)^T.
\end{eqnarray}

\noindent Thus Expression \ref{eq:det_of_blocks} is equivalent to

\begin{equation}
\label{eq:simplified_determinant}
\prod_{i=1}^p \det \left( I_{-i}^T R_i^T \cdots R_1^T \partial_{i} Y_i \right).
\end{equation}

\noindent Now consider the $k,l$ element of the $(n-i) \times (n-i)$ block matrix $I_{-i}^T R_i^T \cdots R_1^T \partial_{i} Y_i $. This can be written as 

\begin{eqnarray}
\label{eq:kl_element}
e_{i+k}^T R_i^T \cdots R_1^T \partial_{i,i+l} Y_i &=&  e_{i+k}^T R_i^T \cdots R_1^T \partial_{i,i+l} (R_1 \cdots R_i e_i)\nonumber \\ \nonumber
&=&  e_{i+k}^T R_i^T \cdots R_1^T R_1 \cdots R_{i-1} (\partial_{i,i+l} R_i e_i)\nonumber \\ 
&=&  e_{i+k}^T R_i^T  (\partial_{i,i+l} R_i e_i).
\end{eqnarray}

\noindent Since  $e_{i+k}^T R_i^T R_i e_i =0$, taking the derivatives of both sides and applying the product rule yields

\begin{eqnarray}
\label{eq:kl_element_2}
&&\partial_{i,i+l} (e_{i+k}^T R_i^T R_i e_i) = \partial_{i,i+l} 0\nonumber \\
&\Rightarrow& (\partial_{i,i+l} e_{i+k}^T R_i^T) R_i e_i + e_{i+k}^T R_i^T ( \partial_{i,i+l}R_i e_i) = 0\nonumber \\
&\Rightarrow& e_{i+k}^T R_i^T  (\partial_{i,i+l}R_i e_i) = -(\partial_{i,i+l} e_{i+k}^T R_i^T) R_i e_i.
\end{eqnarray}

\noindent Combining expression \ref{eq:kl_element_2} this fact with expression \ref{eq:kl_element}, the expression for the $k,l$ element of $I_{-i}^T R_i^T \cdots R_1^T \partial_{i} Y_i $ becomes $-(\partial_{i,i+l} e_{i+k}^T R_i^T) R_i e_i$.

\noindent However, note that

\begin{eqnarray}
e_{i+k}^T R_i^T &=&  e_{i+k}^T R_{in}^T \cdots R_{i,i+1}^T = e_{i+k}^T R_{i,i+k}^T \cdots R_{i,i+1}^T,
\end{eqnarray}

\noindent and the partial derivative of this expression with respect to $i,i+l$ is zero when $k > l$. Thus it is apparent that $I_{-i}^T R_i^T \cdots R_1^T \partial_{i} Y_i $ contains zeros above the diagonal and that $\det \left( I_{-i}^T R_i^T \cdots R_1^T \partial_{i} Y_i \right)$ is simply the product of the diagonal elements of the matrix.

%%%%%%%%%%%%%%%%%%%%%%%%%
\subsection{Diagonal Elements of the Block Matrices}
%%%%%%%%%%%%%%%%%%%%%%%%%
To obtain the diagonal terms of the block matrices, we directly compute $-\partial_{i,i+l} e_{i+k}^T R_i^T$ for $l=k$, $R_i e_i$, and their inner-product. Defining $D_{ij} := \partial_{ij} R_{ij}$,

\begin{eqnarray}
-\partial_{i,i+k} R_i e_{i+k} &=&   -\partial_{i,i+k} (R_{i,i+1} \cdots R_{i,i+k} e_{i+k}) \\
&=& -R_{i,i+1} \cdots R_{i,i+k-1} D_{i,i+k} e_{i+k} \\
\\
&=&
R_{i,i+1} \cdots R_{i,i+k-1}
\begin{pmatrix}
0\\
\vdots\\
0\\
\cos \theta_{i,i+k}\\
0\\
\vdots\\
0\\
\sin \theta_{i,i+k}\\
0\\
\vdots\\
0
\end{pmatrix}
\\
&=&
R_{i,i+1} \cdots R_{i,i+k-2}
\begin{pmatrix}
0\\
\vdots\\
0\\
\cos \theta_{i,i+k-1} \cos \theta_{i,i+k}\\
0\\
\vdots\\
0\\
\sin \theta_{i,i+k-1} \cos \theta_{i,i+k}\\
\sin \theta_{i,i+k}\\
0\\
\vdots\\
0
\end{pmatrix}
\\
&=&
\begin{pmatrix}
0\\
\vdots\\
0\\
\cos \theta_{i,i+1} \cos \theta_{i,i+2} \cdots \cos \theta_{i,i+k-1} \cos \theta_{i,i+k}\\
\sin \theta_{i,i+1} \cos \theta_{i,i+2} \cdots \cos \theta_{i,i+k-1} \cos \theta_{i,i+k}\\
\vdots\\
\sin \theta_{i,i+k-1} \cos \theta_{i,i+k}\\
\sin \theta_{i,i+k}\\
0\\
\vdots\\
0
\end{pmatrix}
\\
\end{eqnarray}

\noindent which is zero up to the $i$th spot. After the $i+k$th spot,

\begin{eqnarray}
R_i e_i &=&   R_{i,i+1} \cdots R_{in} e_i \\
\\
&=&
\begin{pmatrix}
0\\
\vdots\\
0\\
\cos \theta_{i,i+1} \cos \theta_{i,i+2} \cdots \cos \theta_{i,n-1} \cos \theta_{in}\\
\sin \theta_{i,i+1} \cos \theta_{i,i+2} \cdots \cos \theta_{i,n-1} \cos \theta_{in}\\
\vdots\\
\sin \theta_{i,n-1} \cos \theta_{in}\\
\sin \theta_{in}
\end{pmatrix}.
\end{eqnarray}

\noindent Finally, directly computing the inner-product of $-\partial_{i,i+l} e_{i+k}^T R_i^T$ and $R_i e_i$ yields

\begin{eqnarray}
-(\partial_{i,i+l} e_{i+k}^T R_i^T) (R_i e_i)
&=&
\cos^2 \theta_{i,i+1} \cos^2 \theta_{i,i+2}  \cdots \cos^2 \theta_{i,i+k}  \cos \theta_{i,i+k+1} \cdots \cos \theta_{in} \nonumber\\
&+& \sin^2 \theta_{i,i+1} \cos^2 \theta_{i,i+2} \cdots \cos^2 \theta_{i,i+k}  \cos \theta_{i,i+k+1} \cdots \cos \theta_{in} \nonumber\\
&+&  \sin^2 \theta_{i,i+2} \cos^2 \theta_{i,i+3} \cdots \cos^2 \theta_{i,i+k}  \cos \theta_{i,i+k+1} \cdots \cos \theta_{in} \nonumber\\
&\vdots&  \nonumber\\
&+&  \sin^2 \theta_{i,i+k} \cos \theta_{i,i+k+1} \cdots \cos \theta_{in}
 \nonumber\\
&=&
\cos^2 \theta_{i,i+2} \cos^2 \theta_{i,i+3}  \cdots \cos^2 \theta_{i,i+k}  \cos \theta_{i,i+k+1} \cdots \cos \theta_{in}  \nonumber\\
&+&  \sin^2 \theta_{i,i+2} \cos^2 \theta_{i,i+3} \cdots \cos^2 \theta_{i,i+k}  \cos \theta_{i,i+k+1} \cdots \cos \theta_{in}  \nonumber\\
&\vdots&  \nonumber\\
&+&  \sin^2 \theta_{i,i+k} \cos \theta_{i,i+k+1} \cdots \cos \theta_{in}  \nonumber\\
&=& \cdots \nonumber\\
&=& \cos \theta_{i,i+k+1} \cdots \cos \theta_{in}\nonumber\\
&=& \prod_{k=i+1}^n \cos \theta_{ik}.
\end{eqnarray}

\noindent Thus the determinant of the entire block matrix $I_{-i}^T R_i^T \cdots R_1^T \partial_{i} Y_i $ simplifies to

\begin{equation}
\prod_{k=i+1}^n \left( \prod_{j=k+1}^n \cos \theta_{ik} \right) = \prod_{j=i+1}^n \cos^{j-i-1} \theta_{ij}.
\end{equation}

\noindent Combining this with Expression \ref{eq:simplified_determinant} yields

\begin{eqnarray}
\prod_{i=1}^p \det \left( I_{-i}^T R_i^T \cdots R_1^T \partial_{i} Y_i \right) = \prod_{i=1}^p \prod_{j=i+1}^n \cos^{j-i-1} \theta_{ij}.
\end{eqnarray}

%%%%%%%%%%%%%%%%%%%%%%%%%
%%%%%%%%%%%%%%%%%%%%%%%%%
%%%%%%%%%%%%%%%%%%%%%%%%%
\bibliographystyle{ba}
\bibliography{sample}



\end{document}

